\RequirePackage{plautopatch}
\documentclass[14pt,aspectratio=169,xcolor=dvipsnames,table,dvipdfmx]{beamer}
\usepackage{bxdpx-beamer} % dvipdfmxなので必要

%Beamerの設定
\usetheme{Boadilla}

%Beamerフォント設定
\usefonttheme{professionalfonts} % Be professional!
\usepackage[T1]{fontenc}
\usepackage{mlmodern}  % 太いComputer Modern
% MLmodernのバグを修正: cf. https://tex.stackexchange.com/questions/646333/size-of-integral-symbol-in-section-header-with-mlmodern
\DeclareFontFamily{OMX}{mlmex}{}
\DeclareFontShape{OMX}{mlmex}{m}{n}{%
   <->mlmex10%
   }{} 
\usepackage{newtxtext} % 数式以外をTXフォントで上書き
\usepackage[deluxe,uplatex]{otf} % 日本語多ウェイト化
\usepackage{physics,bm}
\usepackage{mhchem}
\usepackage{amsmath,amsfonts,amssymb,mathtools,amsthm}
\renewcommand{\familydefault}{\sfdefault}  % 英文をサンセリフ体に
\renewcommand{\kanjifamilydefault}{\gtdefault}  % 日本語をゴシック体に
\usefonttheme{structurebold} % タイトル部を太字
\setbeamerfont{alerted text}{series=\bfseries} % Alertを太字
\setbeamerfont{section in toc}{series=\mdseries} % 目次は太字にしない
\setbeamerfont{frametitle}{size=\Large} % フレームタイトル文字サイズ
\setbeamerfont{title}{size=\LARGE} % タイトル文字サイズ
\setbeamerfont{date}{size=\small}  % 日付文字サイズ

% Babel (日本語の場合のみ・英語の場合は不要)
\uselanguage{japanese}
\languagepath{japanese}
\deftranslation[to=japanese]{Theorem}{定理}
\deftranslation[to=japanese]{Lemma}{補題}
\deftranslation[to=japanese]{Example}{例}
\deftranslation[to=japanese]{Examples}{例}
\deftranslation[to=japanese]{Definition}{定義}
\deftranslation[to=japanese]{Definitions}{定義}
\deftranslation[to=japanese]{Problem}{問題}
\deftranslation[to=japanese]{Solution}{解}
\deftranslation[to=japanese]{Fact}{事実}
\deftranslation[to=japanese]{Proof}{証明}
\def\proofname{証明}

%Beamer色設定
\definecolor{UniBlue}{RGB}{0,150,200} 
\definecolor{AlertOrange}{RGB}{255,76,0}
\definecolor{AlmostBlack}{RGB}{38,38,38}
\setbeamercolor{normal text}{fg=AlmostBlack}  % 本文カラー
\setbeamercolor{structure}{fg=UniBlue} % 見出しカラー
\setbeamercolor{block title}{fg=UniBlue!50!black} % ブロック部分タイトルカラー
\setbeamercolor{alerted text}{fg=AlertOrange} % \alert 文字カラー
\mode<beamer>{
    \definecolor{BackGroundGray}{RGB}{254,254,254}
    \setbeamercolor{background canvas}{bg=BackGroundGray} % スライドモードのみ背景をわずかにグレーにする
}

%フラットデザイン化
\setbeamertemplate{blocks}[rounded] % Blockの影を消す
\useinnertheme{circles} % 箇条書きをシンプルに
\setbeamertemplate{navigation symbols}{} % ナビゲーションシンボルを消す
\setbeamertemplate{footline}[frame number] % フッターはスライド番号のみ

%タイトルページ
\setbeamertemplate{title page}{%
    \vspace{2.5em}
    {\usebeamerfont{title} \usebeamercolor[fg]{title} \inserttitle \par}
    {\usebeamerfont{subtitle}\usebeamercolor[fg]{subtitle}\insertsubtitle \par}
    \vspace{1.5em}
    \begin{flushright}
        \usebeamerfont{author}\insertauthor\par
        \usebeamerfont{institute}\insertinstitute \par
        \vspace{3em}
        \usebeamerfont{date}\insertdate\par
        \usebeamercolor[fg]{titlegraphic}\inserttitlegraphic
    \end{flushright}
}

% Algorithm系
\usepackage{algorithm}
\usepackage[noend]{algorithmic}
\algsetup{linenosize=\color{fg!50}\footnotesize}
\renewcommand\algorithmicdo{:}
\renewcommand\algorithmicthen{:}
\renewcommand\algorithmicrequire{\textbf{Input:}}
\renewcommand\algorithmicensure{\textbf{Output:}}

% 定理
\theoremstyle{definition}
\newenvironment{mythm}{\begin{alertblock}{定理}}{\end{alertblock}} %自分の結果は赤色で表示

\AtBeginSection[]{
    \frame{\tableofcontents[currentsection, hideallsubsections]} %目次スライド
}
\usepackage[absolute,overlay]{textpos}
\usepackage{array} % needed for \arraybackslash
\usepackage{adjustbox} % for \adjincludegraphics


%タイトル
\title{J.W.Negele, Nuclear Mean-Field Theory, Physics Today 38, 24(1985)}
\author{\textbf{松本侑真}}
\date{\today}
\institute{}

\begin{document}
\maketitle
\frame{\tableofcontents[hideallsubsections]}

\section{背景}
\begin{frame}{核分裂について}
  \begin{itemize}
    \item 核分裂についての説明
    \item 図とかをいれる
    \item 重い元素と軽い元素どちらでも起こる?\ce{Be^8}の核分裂と\ce{U^{235}}の分裂の仕組みは同じ?
    \item トンネル効果
  \end{itemize}
\end{frame}

\begin{frame}{平均場理論}
  \begin{itemize}
    \item 多体系における相互作用を平均化して一体問題のポテンシャルとして扱う手法
    \item 原子核中では、平均ポテンシャルによって核子の配置が決定(集団運動模型)
    \item 原子核の平均ポテンシャルは、Hartree-Fock近似によって微視的に得られる
  \end{itemize}
  \begin{columns}[t]
    \begin{column}{.5\textwidth}
      \begin{itemize}
        \item $V(r,t)$と核子の波動関数は\\自己無撞着性を持つ
        \item $V(r,t)$の収束解が系の平衡状態と考えられる
      \end{itemize}
    \end{column}
    \begin{column}{.5\textwidth}
      \adjincludegraphics[width=1\linewidth,valign=t]{heikinba.png}
    \end{column}
  \end{columns}
\end{frame}
\begin{frame}{Hartree-Fock近似}
  \begin{itemize}
    \item 時間依存のHF方程式(TDHF方程式)は比較的簡単に扱える第一原理的な計算手法
  \end{itemize}
  \begin{block}{TDHFの問題点}
    \begin{itemize}
      \item 多体波動関数が1つのスレーター行列式で表されるという近似を用いている
            \begin{itemize}
              \item 複数のスレーター行列式の線形結合で波動関数を表すことで、より正確な状態が得られることが知られている
            \end{itemize}
      \item 重い原子核の核分裂反応を正しく記述できない
            \begin{itemize}
              \item トンネル効果などの量子効果を正確に記述出来ていない
            \end{itemize}
    \end{itemize}
  \end{block}
\end{frame}


\section{目的}
\begin{frame}{平均場理論の枠組みで核分裂反応を記述する}
  \begin{block}{経路積分を使う理由}
    \begin{itemize}
      \item トンネル効果を含む量子論の現象を記述できる
      \item 鞍点法(Stationary Phase Approximation:SPA)で近似ができる
    \end{itemize}
  \end{block}
  \begin{columns}[t]
    \begin{column}{.53\textwidth}
      \vspace{-5mm}
      \begin{exampleblock}{経路積分のイメージ}
        \begin{itemize}
          \item 状態の遷移確率振幅が経路積分で与えられる。
          \item 始状態$(q_i,t_i)$から終状態$(q_f,t_f)$\\に至るあらゆる経路が寄与する
          \item 作用の変分が$\delta S[q]=0$となる経路$q(t)$からの寄与が大きい
        \end{itemize}
      \end{exampleblock}
    \end{column}
    \begin{column}{.44\textwidth}
      \adjincludegraphics[width=1\linewidth,valign=t]{keiro.png}
    \end{column}
  \end{columns}

\end{frame}

\begin{frame}{経路積分の数式表現}
  \begin{itemize}
    \item 時間発展演算子$U(t,t_0)=e^{-i\hat{H}(t-t_0)/\hbar}$
    \item 始状態を$\ket{\psi_i}$とすると、終状態は$\ket{\psi_f}=U(t,t_0)\ket{\psi_i}$
          \begin{equation*}
            \psi_f(q) = \int dq'\,\bra{q}U(t,t_0)\ket{q'}\bra{q'}\ket{\psi_i}
            = \int dq'\,K(q,q'\,;t,t_0)\psi_i(q')
          \end{equation*}
    \item ファインマン核$K(q_f,q_i\,;t_f,t_i)$の具体形
          \begin{equation*}
            K(q_f,q_i\,;t_f,t_i) = \int\mathcal{D}q\int\mathcal{D}p
            \exp\qty[\frac{i}{\hbar}\int_{t_i}^{t_f}dt\,(p\dot{q}-H(p,q))]
          \end{equation*}
          \begin{equation*}
            \int \mathcal{D}q \coloneqq \prod_{j=1}^{N-1} \int dq_j\;,\quad
            \int \mathcal{D}p \coloneqq \prod_{j=1}^N \int \frac{dp_j}{2\pi\hbar}\;,\quad
            t_f = t_i + N\varDelta t
          \end{equation*}
  \end{itemize}
\end{frame}
\begin{frame}{経路積分の数式表現}
  ポテンシャルが位置にのみ依存する場合、運動量積分が計算できる。
  \begin{align*}
    K(q_f,q_i\,;t_f,t_i) & = \int\mathcal{D}q
    \exp\qty[\frac{i}{\hbar}\int_{t_i}^{t_f}dt\,\qty(\frac{m}{2}\dot{q}^2 - V(q))] \\
                         & = \int\mathcal{D}q\exp\qty[\frac{i}{\hbar}S[q]]
  \end{align*}
  \begin{block}{経路積分の解釈}
    $(q_i,t_i)\to(q_1,t_1)\to\cdots\to(q_{N-1},t_{N-1})\to(q_f,t_f)$
    を経たときの遷移確率振幅を考えて、中間状態に関して全ての経路の和を取ると、\\
    $\ket{q_i\,;t_i}$から$\ket{q_f\,;t_f}$への遷移確率振幅となる。
  \end{block}
\end{frame}

\section{手法}
\begin{frame}{トンネリング解を計算する方法}
  \begin{enumerate}
    \item トンネリング解に対応するエネルギー固有値を求める
          \begin{itemize}
            \item $\trace{(E-\hat{H})^{-1}}=\sum_{n}\ev{(E-E_n)^{-1}}{E_n}$の極がエネルギー固有値
          \end{itemize}
          \begin{align*}
            \trace\frac{1}{E-\hat{H}+i\eta} & =-i\int_0^{\infty}dT\,e^{iET}\int dq\,\bra{q}e^{-i\hat{H}T}\ket{q} \\
                                            & = -i\int_0^{\infty}dT\,e^{iET}\int dq\,\int\mathcal{D}[q]e^{iS[q]}
          \end{align*}
          \begin{itemize}
            \item $q(0)=q(T)$の境界条件(周期運動)を満たした経路積分
            \item あらゆる周期解の中で、$\delta S[q(T)]=0$となるものを足し合わせる
          \end{itemize}
  \end{enumerate}
  \begin{enumerate}
    \setcounter{enumi}{1}
    \item エネルギー固有値に対応する波動関数(核分裂後を記述する状態)を計算する
  \end{enumerate}
\end{frame}

\begin{frame}{エネルギー固有値を求める}
  周期$T$を満たす運動のうち、$\delta S[q(T)]=0$を満たす$q_0(T)$について$W(T)=ET+S[q_0(T)]$とする。SPAを用いると
  \begin{align*}
            & \int_0^{\infty}dT\,e^{iET}\int dq\,\int\mathcal{D}[q]e^{iS[q]}
    \approx\int_0^{\infty}dT\,Ae^{iET + iS[q_0(T)]}                                                                                 \\
    \approx & A\sum_{m=1}^{\infty}f_m e^{iW(T_m)} =A\sum_{m=1}^{\infty}e^{im\pi}\qty(e^{iW(T)})^m = \frac{-Ae^{iW(T)}}{1+e^{iW(T)}}
  \end{align*}
  \begin{itemize}
    \item $\delta S[q_0(T)]=0$であり、$q_0(T)$はEuler-Lagrange方程式を満たす
    \item 周期$T(E)$に対して、$W(mT)=mW(T)$を満たしている
    \item $1+e^{iW(T)}=0\Leftrightarrow W(T)=(2n+1)\pi$はBohr-Sommerfeldの量子化条件に対応
  \end{itemize}
\end{frame}


\begin{frame}{1粒子系での例}
  1つの極小値を持つポテンシャル$V(q)$内を運動する粒子について考える
  \begin{columns}[t]
    \begin{column}{.5\textwidth}
      \begin{itemize}
        \item $m\dv[2]{q_0}{t} = -\grad V(q_0)$
        \item $E-V(q)\geq0$の領域で周期運動
        \item 実際には$W(T)=(2n+1)\pi$を満たす周期$T$の運動が実現する
        \item $T(E)=2\int_{q_1}^{q_2}dq\,\sqrt{\frac{m}{2(E-V(q))}}$\\から$E=E_{n}^{~0}$が求まる
      \end{itemize}
    \end{column}
    \begin{column}{.5\textwidth}
      \adjincludegraphics[width=1\linewidth,valign=t]{pot1.png}
    \end{column}
  \end{columns}
\end{frame}

\begin{frame}{1粒子系での例}
  ここで、$t$を準虚数として、$it\to\tau$という変換を考える。
  \begin{columns}[t]
    \begin{column}{.5\textwidth}
      \begin{itemize}
        \item $m\dv[2]{q_0}{\tau} = -\grad \qty[-V(q_0)]$
        \item 反転したポテンシャル内で$q_2\leq q\leq q_3$を運動しているように見える
        \item 虚時間方向では$\delta S[q_0]=0$が成立しており、SPAでは考慮する必要がある
        \item $E=E_{n}^{~0}$は準安定状態であることが計算からわかる
      \end{itemize}
    \end{column}
    \begin{column}{.5\textwidth}
      \adjincludegraphics[width=1\linewidth,valign=t]{pot2.png}
    \end{column}
  \end{columns}
\end{frame}

\begin{frame}{1粒子系での例}
  虚時間方向での周期運動を考慮して$\trace(E-\hat{H})^{-1}$の計算をする
  \begin{equation*}
    \trace\frac{1}{E-\hat{H}+i\eta} \propto \frac{-2e^{i(W_1+W_3)}-e^{iW_1}-e^{-W_2}-e^{iW_3}}{(1+e^{iW_1})(1+e^{iW_3})+e^{-W_2}}
  \end{equation*}
  \begin{itemize}
    \item $e^{iW_3}$からの寄与は小さく、$1+e^{iW_1(E)}=-e^{-W_2(E)}$を満たす
    \item $E=E_{n}^{~0}+\varDelta E_n$として、補正項$\varDelta E_n$を求めると
          \begin{equation*}
            E = E_{n}^{~0} -\frac{i\Gamma_n}{2}\,,\quad \Gamma_n = 2\frac{\omega(E_n^{~0})}{2\pi}e^{-W_2(E_n^{~0})}
          \end{equation*}
    \item $\Gamma_n$は寿命の逆数であり、準安定状態がトンネル効果で崩壊するWKB近似の式として知られている
  \end{itemize}




\end{frame}

\section{結果}
\begin{frame}{核子多体系に適用する}



\end{frame}

\section{まとめ}

\section{アルゴリズム}
\begin{frame}{アルゴリズムサンプル}
  \begin{block}{Matrix Multiplication}
    \begin{algorithmic}[1]
      \STATE $C = O$
      \FOR{$i = 1, \dots, m$}
      \FOR{$j = 1, \dots, n$}
      \FOR{$k = 1, \dots, r$}
      \STATE $C[i,j] = C[i,j] + A[i, k] \cdot B[k, j]$
      \ENDFOR
      \ENDFOR
      \ENDFOR
      \RETURN $C$
    \end{algorithmic}
  \end{block}
\end{frame}
\end{document}